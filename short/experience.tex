%%% experience.ltx ---

%% Author:        Didier Verna <didier@lrde.epita.fr>
%% Maintainer:    Didier Verna <didier@lrde.epita.fr>
%% Created:       Sun May 23 22:04:41 1999 under XEmacs 21.2 (beta 12)
%% Last Revision: Mon Jul 12 12:03:38 1999


%%% Commentary:

%% Contents management by FCM version 0.1.

\begin{rubric}{E X P \'{E} R I E N C E ~~ P R O F E S S I O N N E L L E}
\subrubric{R\&D:}
\entry*[2014--maintenant]%
  \textbf{Directeur} des relations software et hardware \`{a} \textbf{Aldebaran}, Paris, France,
\entry*[2013--2014]%
  \textbf{Manageur} de l'\'{e}quipe perception \`{a} \textbf{Aldebaran}, Paris, France,
interaction homme/machine, reconnaissance d'objet.
\entry*[2012--maintenant]%
  \textbf{Co-Fondateur} de la \textbf{Fondation OpenCV} et membre du comit\'{e}.
\entry*[2011--2013]%
  \textbf{Ing\'{e}nieur chercheur} \`{a} \textbf{Willow Garage}, Menlo Park,
USA,
en reconnaissance d'objet, SLAM, d\'{e}veloppement d'OpenCV et ROS.
\entry*[2009--2011]%
  \textbf{Ing\'{e}nieur logiciel} \`{a} \textbf{VideoSurf}, San Mateo, USA,
en reconnaissance faciale, r\'{e}sum\'{e} visuel, optimisation de l'analyse
vid\'{e}o.
\entry*[2004--2009]%
  \textbf{Etudiant chercheur} \`{a} University of
California, San Diego, (\textbf{UCSD}), USA, en reconstruction 3D d'objet
non-rigide.
\entry*[2007]
  Contractant pour 4 mois au \textbf{Center for Interdisciplinary Science for Art, Architecture and Archeology} (CISA3): Construction automatique de panoramas tr\`{e}s haute r\'{e}solution.
% \entry*[2005]
%   Contractant pour 4 mois au \textbf{California Institute for
% Telecommunications and Information Technology} (CalIT2): Etude visuelle
% automatique de foules humaines, Projet {\em RESCUE}.
\entry*[2003]
  Stage de 8 mois \`{a} University of California, San Diego,  (\textbf{UCSD}), USA : Etude visuelle automatique du comportement animal, Projet {\em Smart Vivarium}.
\entry*[2002]
  Stage de 6 mois au \textbf{CNES}: conception d'un logiciel de simulation de cr\'{e}ation de d\'{e}bris et d'une m\'{e}thode de d\'{e}tection de collision avec des d\'{e}bris au lancement.
\entry*[2002]
  Stage de 3 mois \`{a} l'\textbf{ONERA} dans le D\'{e}partement de M\'{e}canique du
Vol: \'e{tude} de l'optimisation du remorquage d'un planneur.
% Teaching Experience
% \subrubric{Enseignement:}
% \entry*[2007]
%   Charg\'{e} de cours \`{a} UCSD pour les classe de \emph{Reconnaissance d'Objet} et de \emph{Traitement d'Image}.
% \entry*[2006]
%   Pr\'{e}paration d'un cours d'utilisation efficace de Matlab (sous l'\'{e}gide de Drs. Serge Belongie et Yoav Freund).
% Dev
\subrubric{D\'{e}veloppement:}
\entry*[OpenCV]
d\'{e}veloppement et gestion de l'\'{e}quipe OpenCV.
\entry*[ROS]
d\'{e}veloppement et maintien d'une cinquantaine de paquetages.
\entry*[Tracking]
{\em Surveillance Video Entertainment System}, (SVEN): logiciel de tracking humain en temps r\'{e}el, avec 
un descripteur d'apparence et une analyse d'expression de visages.
\entry*[Panorama]
{\em da Vinci Code}: logiciel rapide pour fusionner des images en un panorama haute r\'{e}solution (Gigapixel) avec 
autocalibration.
\entry*[Toolbox]
{\em Vincent's Structure from Motion Toolbox for Matlab}: module pour Matlab incluant plusieurs routines de reconstruction 3D (pour un objet rigide, non-rigide, ajustement de faisceaux, calcul d'orientation \ldots).
\end{rubric}

%%% Local Variables:
%%% TeX-master: cv.ltx
%%% ispell-local-dictionary: "francais"
%%% TeX-master: t
%%% End:

%%% experience.ltx ends here
