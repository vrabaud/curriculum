\documentstyle[resume]{article}
\begin{document}

\name{\bf VINCENT RABAUD}
\addresses
% {UCSD/CSE-EBU3B Rm.~4150\\
% 9500 Gilman Drive, Dept.~0404\\
% La Jolla, CA 92093-0404}
{218 av du Maine\\
Paris, 75014\\
France}
{
06 95 57 25 85 French phone\\
(858) 437-3417 US phone\\
vincent.rabaud@gmail.com}

\begin{llist}
\sectiontitle{Education}

\employer{UNIVERSITY OF CALIFORNIA, SAN DIEGO} \location{San Diego, CA}
Ph.D. in Computer Science.\\
Dissertation:\textit{Manifold Learning Techniques for Non-Rigid Structure from
Motion}.\\
Advisor: Serge Belongie.\\
Research Interests: Structure from Motion, Multiview Geometry, Panorama, Image Manifold Learning, Tracking, Behavior Analysis, Optimization, Visual Captchas, Perception.

\employer{SUPAERO} \location{Toulouse, France}
M.S. in Aeronautical and Space Engineering, Space Imagery Major, 2003.

\employer{ECOLE POLYTECHNIQUE} \location{Paris, France}
B.S./M.S. in Applied Math, Fluid Mechanics and Parallel Computing, 2001.

% Research Experience
\sectiontitle{Work Experience}
\employer{ALDEBARAN ROBOTICS}\location{Paris, France}
\dates{April 2013--present}
Perception Team Manager, Object Recognition (2d/3d),

\employer{WILLOW GARAGE}\location{Menlo Park, CA} 
\dates{January 2011--March 2013}
Research Engineer, Object Recognition (2d/3d), OpenCV developement and management, SLAM, ROS infrastructure, 

\employer{VIDEOSURF}\location{San Mateo, CA} 
\dates{March 2009--January 2011}
Software Engineer, Face Recognition, Video Summary, Video Pipeline
Optimizations.

\employer{UNIVERSITY OF CALIFORNIA, SAN DIEGO}\location{La Jolla, CA} 
\dates{January 2004--March 2009}
Graduate Student Researcher, Department of Computer Science and Engineering.

\employer{CENTER FOR INTERDISCIPLINARY SCIENCE FOR ART, ARCHITECTURE AND ARCHAEOLOGY (CISA3)}\location{La Jolla, CA} 
\dates{June 2007--September 2007}
Intern, Painting Panoramas, High Resolution Mosaics.

\employer{CALIT2}\location{La Jolla, CA} 
\dates{June 2005--September 2005}
Intern, Visual Crowd Management, {\em RESCUE} Project.

\employer{UNIVERSITY OF CALIFORNIA, SAN DIEGO}\location{La Jolla, CA} 
\dates{May 2003--December 2004}
Intern, Animal Behavior Analysis, {\em Smart Vivarium} Project.

\employer{CENTRE NATIONAL D'ETUDES SPATIALES (CNES)}\location{Toulouse, France} 
\dates{Summer and Fall 2002}
Intern, Space Mechanics Department, French Space Agency.

\employer{OFFICE NATIONAL D'ETUDES ET DE RECHERCHES AEROSPATIALES (ONERA)} \location{Toulouse, France}
\dates{Spring 2002}
Intern, Flight Mechanics Department.

\employer{DYNAFLOW-INC} \location{Jessup, MD}
\dates{Spring 2001}
Intern, Fluid Mechanics Modeling.

% Teaching Experience
\sectiontitle{Teaching Experience}
\employer{UNIVERSITY OF CALIFORNIA, SAN DIEGO}\location{La Jolla, CA}
CSE252C, {\em Object Recognition}, Fall 2007: Teaching Assistant\\
CSE166, {\em Image Processing}, Fall 2007: Teaching Assistant

% Professional Activities
\sectiontitle{Professional Activities}

Treasurer and member of the board of the OpenCV foundation.

Reviewer: IEEE International Conference on Computer Vision, IEEE Conference on Computer Vision and Pattern Recognition, 
SIGGRAPH, IEEE Transaction on Pattern Analysis and Machine Intelligence, International Journal of Computer Vision

Organizer of Pixel-Cafe, the weekly vision and graphics seminar at UCSD.

IEEE member, 2005-present

% Papers
\sectiontitle{Journal Articles}
A.~Ziegler, E.~Christiansen, V.~Rabaud, S.~Belongie, D.~Kriegman, ``In submission'', {\em 
IEEE Transaction on Pattern Analysis and Machine Intelligence} (\textbf{PAMI, in preparation}), 2013.


\sectiontitle{Papers in Reviewed Proceedings}
A.~Ziegler, E.~Christiansen, V.~Rabaud, S.~Belongie, D.~Kriegman, ``Match-time covariance for 
descriptors'', \textbf{BMVC}, 2013.

S.~Leutenegger, P.~T.~Furgale, V.~Rabaud, M.~Chli, K.~Konolige and R.~Siegwart, ``Keyframe-Based Visual-Inertial SLAM using Nonlinear Optimization.
'', (\textbf{RSS}), 2013.

M.~Dimashova, I.~Lysenkov, V.~Rabaud, V.~Eruhimov ``Tabletop Object Scanning with an RGB-D Sensor'' , 3rd Workshop 
on Semantic Perception, \textbf{ICRA}, 2013.

I.~Lysenkov, V.~Rabaud, ``Pose Estimation of Rigid Transparent Objects in Transparent Clutter'' , \textbf{ICRA}, 2013.

E.~Rublee, V.~Rabaud, K.~Konolige and G.~Bradski, ``ORB: an efficient alternative to SIFT or SURF'' , {\em IEEE 
International Conference in Computer Vision}, (\textbf{ICCV}), 2011.

V.~Rabaud and S.~Belongie, ``Linear Embeddings in Non-Rigid Structure from Motion'' , {\em IEEE Conference on Computer 
Vision and Pattern Recognition}, (\textbf{CVPR}), 2009.

V.~Rabaud and S.~Belongie, ``Re-Thinking Non-Rigid Structure From Motion'' , {\em IEEE Conference on Computer Vision and 
Pattern Recognition}, (\textbf{CVPR}), 2008.

S.~Steinbach, V.~Rabaud and S.~Belongie, ``Soylent Grid: it's made of People !'' , {\em Interactive Computer Vision, in 
conjunction with ICCV}, (\textbf{ICV}), 2007.

P.~Doll\'ar, V.~Rabaud and S.~Belongie`, ``Non-Isometric Manifold Learning: Analysis and an Algorithm'', {\em 
International Conference on Machine Learning}, (\textbf{ICML}), 2007.

P.~Doll\'ar, V.~Rabaud and S.~Belongie, ``Learning to Traverse Image Manifolds'' , {\em Neural Information Processing 
Systems}, (\textbf{NIPS}), 2006. 

V.~Rabaud and S.~Belongie, ``Counting Crowded Moving Objects,'', {\em IEEE Conference on Computer Vision and Pattern 
Recognition}, (\textbf{CVPR}), 2006, pp. 705- 711, vol. 1.

P.~Doll\'ar, V.~Rabaud, G.~Cottrell and S.~Belongie, ``Behavior Recognition via Sparse Spatio-Temporal Features,'' {\em 
Joint International Workshop on Visual Surveillance and Performance Evaluation of Tracking and Surveillance}, 
(\textbf{VS-PETS}), 2005. 

S.~Belongie, K.~Branson, P.~Doll\'ar, and V.~Rabaud, ``Monitoring Animal Behavior in the Smart Vivarium,'' {\em 
International Conference on Methods and Techniques in Behavioral Research}, 2005.

V.~Rabaud and S.~Belongie, ``Big Little Icons,'' {\em IEEE Workshop on Computer Vision Applications for the Visually 
Impaired, in conjunction with CVPR}, (\textbf{CVAVI}), 2005.

K.\ Branson, V.\ Rabaud and S.\ Belongie, ``Three Brown Mice: See How They Run,''
{\em Joint International Workshop on Visual Surveillance and Performance Evaluation of Tracking and Surveillance}, 
(\textbf{VSPETS}), 2003, pp.\ 78-85. 

V.\ Rabaud and B.\ Deguine ``A Geometrical Approach To Determine Blackout Windows At Launch,''
AAS/AIAA Space Flight Mechanics Meeting, Ponce, Puerto Rico, (\textbf{AAS}), 2003, 03-187 


% Software
\sectiontitle{Software}
{\em ROS packages}: maintainer of 30 ROS packages and involved in many others.

{\em Recognition Kitchen}: set of tools to develop and execute object recognition.

{\em Surveillance Video Entertainment System}, (SVEN): real-time tracking of pedestrians incorporating appearance 
description, face detection and facial expression analysis.

{\em Painting Panorama}: fast and memory efficient panorama software for very high resolution images of paintings.  Incorporates sparse bundle adjustment, sift and camera auto calibration.

{\em Vincent's Structure from Motion Toolbox for Matlab}: toolbox including many common structure from motion algorithms (e.g. rigid, non-rigid, bundle adjustment, visualization).

% Video
\sectiontitle{Video}
M.~Maschion, V.~Rabaud and S.~Belongie, {\em Computer Vision: Fact and Fiction},
Instructional DVD, 2005.

% Papers
\sectiontitle{Skills}
{\em Computing Platforms}: Unix, Solaris, Clusters (Rocks), Windows, OSX\\
{\em Programming Languages}: C++, Python, Matlab, Javascript, Fortran \\
{\em Programming Libraries}: OpenCV, ROS, Boost, OpenMP, TBB, GStreamer, PVM, MPI, GTK \\
{\em Extra Interests}: Android, Drupal, PHP, MySQL \\
{\em Languages}: French (native), English (fluent), Spanish (fluent), Portuguese (beginner), Italian (beginner)

% Papers
%\sectiontitle{Relevant Coursework}
%Computer Vision I (Image Analysis, Radiometry, SFM)\\
%Computer Vision II (Multiple View Geometry)\\
%Computer Vision III (Object Recognition)\\
%Machine Learning\\
%Graphical Models

\sectiontitle{References}

% \textbf{Prof. Sanjoy Dasgupta}\\
% University of California, San Diego – Computer Science \& Engineering\\
% CSE-EBU3B 4138; 9500 Gilman Dr.; \#0404 La Jolla, CA 92093-0404, USA\\
% dasgupta@cs.ucsd.edu

% \textbf{Maurizio Seracini}\\
% Editech\\
% Via Dei Bardi, 28\\
% 50125 Firenze (FI), Italy

\textbf{Prof. Serge Belongie}\\
Cornell University\\
111 Eighth Avenue \#302, New York, NY 10011, USA\\
sjb344@cornell.edu

\textbf{Dr. Gary Bradski}\\
Willow Garage, Inc. \\
68 Willow Road, Menlo Park, CA 94025, USA\\
gary@industrial-perception.com

\textbf{Dr. Kurt Konolige}\\
Industrial Perception, Inc. \\
911 Industrial Ave, Palo Alto, CA 94303, USA\\
kurt@industrial-perception.com

\textbf{Dr. Brian Gerkey}\\
Open Source Robotics Foundation\\
419 N Shoreline Blvd, Mountain View, CA 94043, USA\\
gerkey@osrfoundation.org

\textbf{Prof. David Kriegman}\\
University of California, San Diego – Computer Science \& Engineering\\
CSE-EBU3B 4120; 9500 Gilman Dr.; \#0404 La Jolla, CA 92093-0404, USA\\
kriegman@cs.ucsd.edu


\end{llist}

%{\em Last update: \today}
{\em Last update: \today}

\end{document}
