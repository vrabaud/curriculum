\documentclass{article}

\usepackage[french,english]{babel}
\usepackage{iflang}
\usepackage{resume}

\begin{document}

\selectlanguage{english}

\name{\bf VINCENT RABAUD}

\begin{llist}
\sectiontitle{Education}

\employer{UNIVERSITY OF CALIFORNIA, SAN DIEGO} \location{San Diego, CA, USA}
\IfLanguageName{english}
{
Ph.D. in Computer Science, 2009\\
Dissertation:\textit{Manifold Learning Techniques for Non-Rigid Structure from
Motion}.\\
Advisor: Serge Belongie.\\
Research Interests: Structure from Motion, Multiview Geometry, Panorama, Image Manifold Learning, Tracking, Behavior Analysis, Optimization, Visual Captchas, Perception.
}
{
Ph.D. en Informatique, 2009\\
Dissertation:\textit{Manifold Learning Techniques for Non-Rigid Structure from
Motion}.\\
Centres d'int\'{e}r\^{e}t: Reconstruction 3D par \'{e}tude de mouvement, Segmentation de mouvement, Tracking, 
Apprentissage de vari\'{e}t\'{e}s, Perception visuelle.\\
Ma\^{i}tre de th\`{e}se: Serge Belongie.
}

\employer{SUPAERO} \location{Toulouse, France}
\IfLanguageName{english}
{
M.S. in Aeronautical and Space Engineering, Space Imagery Major, 2003.
}
{
Dipl\^{o}me d'ing\'{e}nieur, 2003
}

\employer{ECOLE POLYTECHNIQUE} \location{Paris, France}
\IfLanguageName{english}
{
B.S./M.S. in Applied Math, Fluid Mechanics and Parallel Computing, 2001.
}
{
Dipl\^{o}me d'ing\'{e}nieur, 2001
}

% Research Experience
\IfLanguageName{english}
{
\sectiontitle{Work Experience}
}
{
\sectiontitle{Exp\'{e}rience Professionnelle}
}

\employer{ALDEBARAN}\location{Paris, France}
\dates{09/2014--}
\IfLanguageName{english}
{
SW/HW link director
}
{
Directeur des relations software/hardware
}

\employer{ALDEBARAN}\location{Paris, France}
\dates{04/2013--09/2013}
\IfLanguageName{english}
{
Perception Team Manager, Object Recognition (2d/3d), Human Interaction.
}
{
Manageur de l'\'{e}quipe Perception. Interaction Audio/Video, Reconnaissance d'Objets (2d/3d), Reconaissance/Tracking de 
Personne.
}

\employer{OPENCV FOUNDATION\\}
\dates{06/2012--}
\IfLanguageName{english}
{
Co-Founder and Member of the Board.
}
{
Co-Fondateur et Membre du Conseil.
}

\employer{WILLOW GARAGE}\location{Menlo Park, CA, USA} 
\dates{01/2011--03/2013}
\IfLanguageName{english}
{
Research Engineer, Object Recognition (2d/3d), OpenCV developement and management, SLAM, ROS infrastructure.
}
{
Ing\'{e}nieur Chercheur, Reconnaissance d'objets (2d/3d), d\'{e}veloppement et management gestion de l'\'{e}quipe 
OpenCV, SLAM, d\'{e}veloppement et maintien de ROS.
}

\employer{VIDEOSURF}\location{San Mateo, CA, USA} 
\dates{03/2009--01/2011}
\IfLanguageName{english}
{
Software Engineer, Face Recognition, Video Summary, Video Pipeline
Optimizations.
}
{
Ing\'{e}nieur D\'{e}veloppeur, Reconnaissance faciale, Analyse
Vid\'{e}o.
}

\employer{UNIVERSITY OF CALIFORNIA, SAN DIEGO}\location{La Jolla, CA, USA} 
\dates{01/2004--03/2009}
\IfLanguageName{english}
{
Graduate Student Researcher, Department of Computer Science and Engineering.
}
{
Etudiant chercheur, D\'{e}partment d'informatique.
}

\employer{CENTER FOR INTERDISCIPLINARY SCIENCE FOR ART, ARCHITECTURE AND ARCHAEOLOGY (CISA3)}\location{La Jolla, CA, 
USA} 
\dates{06/2007--09/2007}
\IfLanguageName{english}
{
Intern, Painting Panoramas, High Resolution Mosaics.
}
{
Stagiaire, Reconstruction automatique de panorama de peinture, Mosa\"{\i}que haute r\'{e}solution.
}

\employer{CALIT2}\location{La Jolla, CA, USA} 
\dates{06/2005--09/2005}
\IfLanguageName{english}
{
Intern, Visual Crowd Management, {\em RESCUE} Project.
}
{
Stagiaire, Etude visuelle de foules humaines, Projet {\em RESCUE}.
}

\employer{UNIVERSITY OF CALIFORNIA, SAN DIEGO}\location{La Jolla, CA, USA} 
\dates{05/2003--12/2004}
\IfLanguageName{english}
{
Intern, Animal Behavior Analysis, {\em Smart Vivarium} Project.
}
{
Stagiaire, Groupe de vision par ordinateur, Projet {\em Smart Vivarium}.
}

\employer{CENTRE NATIONAL D'ETUDES SPATIALES (CNES)}\location{Toulouse, France} 
\dates{06/2002--12/2002}
\IfLanguageName{english}
{
Intern, Space Mechanics Department, French Space Agency.
}
{
Stagiaire, D\'{e}partement de M\'{e}canique Spatiale.
}

\employer{OFFICE NATIONAL D'ETUDES ET DE RECHERCHES AEROSPATIALES (ONERA)} \location{Toulouse, France}
\dates{03/2002--06/2002}
\IfLanguageName{english}
{
Intern, Flight Mechanics Department.
}
{
Stagiare, D\'{e}partement de M\'{e}canique du Vol.
}

\employer{DYNAFLOW-INC} \location{Jessup, MD}
\dates{04/2001--06/2001}
\IfLanguageName{english}
{
Intern, Fluid Mechanics Modeling.
}
{
Stagiaire, Modelisation de M\'{e}canique des Fluides.
}

% Teaching Experience
\IfLanguageName{english}
{
\sectiontitle{Teaching Experience}
}
{
\sectiontitle{Exp\'{e}rience Educative}
}
\employer{UNIVERSITY OF CALIFORNIA, SAN DIEGO}\location{La Jolla, CA, USA}
\IfLanguageName{english}
{
CSE252C, {\em Object Recognition}, Fall 2007: Teaching Assistant\\
CSE166, {\em Image Processing}, Fall 2007: Teaching Assistant
}
{
SE252C, {\em Reconnaissance d'objet}, Automne 2007: Assistant de Professeur\\
CSE166, {\em Traitement d'image}, Automne 2007: Assistant de Professeur
}

% Professional Activities
\IfLanguageName{english}
{
\sectiontitle{Professional Activities}

Co-founder and member of the board of the OpenCV foundation.

Mentor and organizer of Google Summer of Code for OpenCV from 2011 to 2015.

Reviewer: IEEE International Conference on Computer Vision, IEEE Conference on Computer Vision and Pattern Recognition, 
SIGGRAPH, IEEE Transaction on Pattern Analysis and Machine Intelligence, International Journal of Computer Vision

Organizer of Pixel-Cafe, the weekly vision and graphics seminar at UCSD.

IEEE member, 2005-present
}
{
\sectiontitle{Activit\'{e} Professionnelle}

Tr\'{e}sorier et membre du conseil d'administration de la fondation OpenCV.

Mentor et organisateur de Google Summer of Code pour OpenCV de 2011 \`{a} 2015.

Membre de comit\'{e} de lecture: IEEE International Conference on Computer Vision, IEEE Conference on Computer Vision 
and Pattern Recognition, SIGGRAPH.

Organisateur du Pixel-Cafe, le s\'{e}minaire hebdomadaire de vision et image par ordinateur \`{a} UCSD.

Membre de IEEE.
}

% Skills
\IfLanguageName{english}
{
\sectiontitle{Skills}
{\em Skills}: Vision (SfM, web video analysis, image retrieval, object recognition), Robotics, Management \\
{\em Programming Languages}: C++, Python\\
{\em Programming Libraries}: OpenCV, ROS, Boost, OpenMP, TBB, PVM, MPI \\
{\em Extra Interests}: Android, Drupal, Matlab, Javascript, PHP, MySQL, Fortran \\
{\em Languages}: French (native), English (fluent), Spanish (fluent), Portuguese (beginner), Italian (beginner)
}
{
\sectiontitle{Comp\'{e}tences}
{\em Comp\'{e}tences}: Vision, Robotique, Management\\
{\em Langages}: C++, Python, exp\'{e}rience en Matlab, Javascript, Fortran \\
{\em Librairies}: OpenCV, ROS, Boost, OpenMP, TBB, PVM, MPI \\
{\em Autres Int\'{e}r\^{e}ts}: Android, Drupal, Matlab, Javascript, PHP, MySQL, Fortran \\
{\em Langues}: Fran\c{c}ais (natif), Anglais (fluent), Espagnol (fluent), Portugais (d\'{e}butant), Italien 
(d\'{e}butant)
}

% Software
\IfLanguageName{english}
{
\sectiontitle{Software}
{\em ROS packages}: maintainer/developer of 60+ ROS packages and involved in the core development.

{\em Recognition Kitchen}: set of tools to develop and execute object recognition.

{\em Surveillance Video Entertainment System}, (SVEN): real-time tracking of pedestrians incorporating appearance 
description, face detection and facial expression analysis.

{\em Painting Panorama}: fast and memory efficient panorama software for very high resolution images of paintings.  Incorporates sparse bundle adjustment, sift and camera auto calibration.

{\em Vincent's Structure from Motion Toolbox for Matlab}: toolbox including many common structure from motion algorithms 
(e.g. rigid, non-rigid, bundle adjustment, visualization).
}
{
\sectiontitle{Logiciels}
{\em ROS packages}: mainteneur et d\'{e}veloppeur de 60+ paquet ROS et impliqu\'{e} dans le d\'{e}veloppement du 
noyau.

{\em Recognition Kitchen}: ensemble d'outils pour d\'{e}velopper et faire de la reconnaissance d'objet.

{\em Surveillance Video Entertainment System}, (SVEN): logiciel de tracking en temps r\'eel de personnes, incluant un 
descripteur d'apparence, une d\'{e}tection de visage et une analyse d'expression.

{\em Painting Panorama}: logiciel rapide pour fusionner des images en une mosa\"{i}que haute r\'{e}solution.  L'accent 
a \'{e}t\'{e} mis sur l'efficacit\'{e} et la n\'{e}cessit\'{e} de faibles ressources.  Ce logiciel inclut un ajustement 
de faisceaux, les descripteurs SIFT et une calibration automatique.

{\em Vincent's Structure from Motion Toolbox for Matlab}: toolbox pour Matlab incluant plusieurs routines de 
reconstruction 3D (pour un objet rigide, non-rigide, ajustement de faisceaux, calcul d'orientation \dots).
}

% Papers
\IfLanguageName{english}
{
\sectiontitle{Journal Articles}
}
{
\sectiontitle{Articles dans des Revues Internationales}
}
A.~Ziegler, E.~Christiansen, V.~Rabaud, S.~Belongie, D.~Kriegman, ``In submission'', {\em 
IEEE Transaction on Pattern Analysis and Machine Intelligence} (\textbf{PAMI, in preparation}), 2013.


\IfLanguageName{english}
{
\sectiontitle{Papers in Reviewed Proceedings}
}
{
\sectiontitle{Articles dans des Conf\'{e}rences Internationales}
}
A.~Ziegler, E.~Christiansen, V.~Rabaud, S.~Belongie, D.~Kriegman, ``Match-time covariance for 
descriptors'', \textbf{BMVC}, 2013.

S.~Leutenegger, P.~T.~Furgale, V.~Rabaud, M.~Chli, K.~Konolige and R.~Siegwart, ``Keyframe-Based Visual-Inertial SLAM using Nonlinear Optimization.
'', (\textbf{RSS}), 2013.

M.~Dimashova, I.~Lysenkov, V.~Rabaud, V.~Eruhimov ``Tabletop Object Scanning with an RGB-D Sensor'' , 3rd Workshop 
on Semantic Perception, \textbf{ICRA}, 2013.

I.~Lysenkov, V.~Rabaud, ``Pose Estimation of Rigid Transparent Objects in Transparent Clutter'' , \textbf{ICRA}, 2013.

E.~Rublee, V.~Rabaud, K.~Konolige and G.~Bradski, ``ORB: an efficient alternative to SIFT or SURF'' , {\em IEEE 
International Conference in Computer Vision}, (\textbf{ICCV}), 2011.

V.~Rabaud and S.~Belongie, ``Linear Embeddings in Non-Rigid Structure from Motion'' , {\em IEEE Conference on Computer 
Vision and Pattern Recognition}, (\textbf{CVPR}), 2009.

V.~Rabaud and S.~Belongie, ``Re-Thinking Non-Rigid Structure From Motion'' , {\em IEEE Conference on Computer Vision and 
Pattern Recognition}, (\textbf{CVPR}), 2008.

S.~Steinbach, V.~Rabaud and S.~Belongie, ``Soylent Grid: it's made of People !'' , {\em Interactive Computer Vision, in 
conjunction with ICCV}, (\textbf{ICV}), 2007.

P.~Doll\'ar, V.~Rabaud and S.~Belongie`, ``Non-Isometric Manifold Learning: Analysis and an Algorithm'', {\em 
International Conference on Machine Learning}, (\textbf{ICML}), 2007.

P.~Doll\'ar, V.~Rabaud and S.~Belongie, ``Learning to Traverse Image Manifolds'' , {\em Neural Information Processing 
Systems}, (\textbf{NIPS}), 2006. 

V.~Rabaud and S.~Belongie, ``Counting Crowded Moving Objects,'', {\em IEEE Conference on Computer Vision and Pattern 
Recognition}, (\textbf{CVPR}), 2006, pp. 705- 711, vol. 1.

P.~Doll\'ar, V.~Rabaud, G.~Cottrell and S.~Belongie, ``Behavior Recognition via Sparse Spatio-Temporal Features,'' {\em 
Joint International Workshop on Visual Surveillance and Performance Evaluation of Tracking and Surveillance}, 
(\textbf{VS-PETS}), 2005. 

S.~Belongie, K.~Branson, P.~Doll\'ar, and V.~Rabaud, ``Monitoring Animal Behavior in the Smart Vivarium,'' {\em 
International Conference on Methods and Techniques in Behavioral Research}, 2005.

V.~Rabaud and S.~Belongie, ``Big Little Icons,'' {\em IEEE Workshop on Computer Vision Applications for the Visually 
Impaired, in conjunction with CVPR}, (\textbf{CVAVI}), 2005.

K.\ Branson, V.\ Rabaud and S.\ Belongie, ``Three Brown Mice: See How They Run,''
{\em Joint International Workshop on Visual Surveillance and Performance Evaluation of Tracking and Surveillance}, 
(\textbf{VSPETS}), 2003, pp.\ 78-85. 

V.\ Rabaud and B.\ Deguine ``A Geometrical Approach To Determine Blackout Windows At Launch,''
AAS/AIAA Space Flight Mechanics Meeting, Ponce, Puerto Rico, (\textbf{AAS}), 2003, 03-187 


% Video
\sectiontitle{Video}
M.~Maschion, V.~Rabaud and S.~Belongie, {\em Computer Vision: Fact and Fiction},
Instructional DVD, 2005.

% Papers
%\sectiontitle{Relevant Coursework}
%Computer Vision I (Image Analysis, Radiometry, SFM)\\
%Computer Vision II (Multiple View Geometry)\\
%Computer Vision III (Object Recognition)\\
%Machine Learning\\
%Graphical Models

\sectiontitle{References}

% \textbf{Prof. Sanjoy Dasgupta}\\
% University of California, San Diego – Computer Science \& Engineering\\
% CSE-EBU3B 4138; 9500 Gilman Dr.; \#0404 La Jolla, CA 92093-0404, USA\\
% dasgupta@cs.ucsd.edu

% \textbf{Maurizio Seracini}\\
% Editech\\
% Via Dei Bardi, 28\\
% 50125 Firenze (FI), Italy

\textbf{Prof. Serge Belongie}\\
Cornell University\\
111 Eighth Avenue \#302, New York, NY 10011, USA\\
sjb344@cornell.edu

\textbf{Dr. Gary Bradski}\\
Willow Garage, Inc. \\
68 Willow Road, Menlo Park, CA 94025, USA\\
gary@industrial-perception.com

\textbf{Dr. Kurt Konolige}\\
Industrial Perception, Inc. \\
911 Industrial Ave, Palo Alto, CA 94303, USA\\
kurt@industrial-perception.com

\textbf{Dr. Brian Gerkey}\\
Open Source Robotics Foundation\\
419 N Shoreline Blvd, Mountain View, CA 94043, USA\\
gerkey@osrfoundation.org

\textbf{Prof. David Kriegman}\\
University of California, San Diego – Computer Science \& Engineering\\
CSE-EBU3B 4120; 9500 Gilman Dr.; \#0404 La Jolla, CA 92093-0404, USA\\
kriegman@cs.ucsd.edu


\end{llist}

{\em Last update: \today}

\end{document}
