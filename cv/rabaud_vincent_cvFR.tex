\documentstyle[resume]{article}
\begin{document}

\name{\bf VINCENT RABAUD}
\addresses
{6 Rue Pierre Bruniere\\
Toulouse, 31000\\
France}
{
06-95-57-25-85\\
vincent.rabaud@gmail.com}

\begin{llist}
\sectiontitle{Formation}

\employer{UNIVERSITY OF CALIFORNIA, SAN DIEGO} \location{San Diego, CA}
PhD. in Computer Science.\\
Centres d'int\'{e}r\^{e}t: Reconstruction 3D par \'{e}tude de mouvement, Segmentation de mouvement, Tracking, Apprentissage de vari\'{e}t\'{e}s, Perception visuelle.\\
Ma\^{i}tre de th\`{e}se: Serge Belongie.

\employer{SUPAERO} \location{Toulouse, France}
Dipl\^{o}me d'ing\'{e}nieur, 2003

\employer{ECOLE POLYTECHNIQUE} \location{Paris, France}
Dipl\^{o}me d'ing\'{e}nieur, 2001

% Research Experience
\sectiontitle{Experience Professionnelle}
\employer{WILLOW GARAGE}\location{Menlo Park, CA}
\dates{Janvier 2011--maintenant}
Ing\'{e}nieur Chercheur, Reconnaissance d'objets (2d/3d), d\'{e}veloppement et management gestion de l'\'{e}quipe 
OpenCV, SLAM, d\'{e}veloppement et maintien de ROS.

\employer{VIDEOSURF}\location{San Mateo, CA}
\dates{Mars 2009--Janvier 2011}
Ing\'{e}nieur D\'{e}veloppeur, Reconnaissance faciale, Analyse
Vid\'{e}o.

\employer{UNIVERSITY OF CALIFORNIA, SAN DIEGO}\location{La Jolla, CA} 
\dates{Janvier 2004--Mars 2009}
Etudiant chercheur, D\'{e}partment d'informatique.

\employer{CENTER FOR INTERDISCIPLINARY SCIENCE FOR ART, ARCHITECTURE AND ARCHEOLOGY (CISA3)}\location{La Jolla, CA} 
\dates{Juin 2007--Septembre 2007}
Stagiaire, Reconstruction automatique de panorama de peinture, Mosa\"{\i}que haute r\'{e}solution.

\employer{CALIT2}\location{La Jolla, CA} 
\dates{Juin 2005--Septembre 2005}
Stagiaire, Etude visuelle de foules humaines, Projet {\em RESCUE}.

\employer{UNIVERSITY OF CALIFORNIA, SAN DIEGO}\location{La Jolla, CA} 
\dates{Mai 2003--Decembre 2004}
Stagiaire, Groupe de vision par ordinateur, Projet {\em Smart Vivarium}.

\employer{CENTRE NATIONAL D'ETUDES SPATIALES (CNES)}\location{Toulouse, France} 
\dates{Et\'{e} et Automne 2002}
Stagiaire, D\'{e}partement de M\'{e}canique Spatiale.

\employer{OFFICE NATIONAL D'ETUDES ET DE RECHERCHES AEROSPATIALES (ONERA)} \location{Toulouse, France}
\dates{Printemps 2002}
Stagiare, D\'{e}partement de M\'{e}canique du Vol.

\employer{DYNAFLOW-INC} \location{Jessup, MD}
\dates{Spring 2001}
Stagiaire, Modelisation de M\'{e}canique des Fluides.

% Teaching Experience
\sectiontitle{Exp\'{e}rience Educative}
\employer{UNIVERSITY OF CALIFORNIA, SAN DIEGO}\location{La Jolla, CA}
CSE252C, {\em Reconnaissance d'objet}, Automne 2007: Assistant de Professeur\\
CSE166, {\em Traitement d'image}, Automne 2007: Assistant de Professeur

% Professional Activities
\sectiontitle{Activit\'{e} Professionnelle}

Tr\'{e}sorier et membre du conseil d'administration de la fondation OpenCV.

Membre de comit\'{e} de lecture: IEEE International Conference on Computer Vision, IEEE Conference on Computer Vision and Pattern Recognition, SIGGRAPH.

Organisateur du Pixel-Cafe, le s\'{e}minaire hebdomadaire de vision et image par ordinateur \`{a} UCSD.

Membre de IEEE.

% Papers
\sectiontitle{Articles dans des Revues Internationales}
A.~Ziegler, E.~Christiansen, V.~Rabaud, S.~Belongie, D.~Kriegman, ``In submission'', {\em 
IEEE Transaction on Pattern Analysis and Machine Intelligence} (\textbf{PAMI, in preparation}), 2013.


\sectiontitle{Articles dans des Conf\'{e}rences Internationales}
A.~Ziegler, E.~Christiansen, V.~Rabaud, S.~Belongie, D.~Kriegman, ``Match-time covariance for 
descriptors'', \textbf{BMVC}, 2013.

S.~Leutenegger, P.~T.~Furgale, V.~Rabaud, M.~Chli, K.~Konolige and R.~Siegwart, ``Keyframe-Based Visual-Inertial SLAM using Nonlinear Optimization.
'', (\textbf{RSS}), 2013.

M.~Dimashova, I.~Lysenkov, V.~Rabaud, V.~Eruhimov ``Tabletop Object Scanning with an RGB-D Sensor'' , 3rd Workshop 
on Semantic Perception, \textbf{ICRA}, 2013.

I.~Lysenkov, V.~Rabaud, ``Pose Estimation of Rigid Transparent Objects in Transparent Clutter'' , \textbf{ICRA}, 2013.

E.~Rublee, V.~Rabaud, K.~Konolige and G.~Bradski, ``ORB: an efficient alternative to SIFT or SURF'' , {\em IEEE 
International Conference in Computer Vision}, (\textbf{ICCV}), 2011.

V.~Rabaud and S.~Belongie, ``Linear Embeddings in Non-Rigid Structure from Motion'' , {\em IEEE Conference on Computer 
Vision and Pattern Recognition}, (\textbf{CVPR}), 2009.

V.~Rabaud and S.~Belongie, ``Re-Thinking Non-Rigid Structure From Motion'' , {\em IEEE Conference on Computer Vision and 
Pattern Recognition}, (\textbf{CVPR}), 2008.

S.~Steinbach, V.~Rabaud and S.~Belongie, ``Soylent Grid: it's made of People !'' , {\em Interactive Computer Vision, in 
conjunction with ICCV}, (\textbf{ICV}), 2007.

P.~Doll\'ar, V.~Rabaud and S.~Belongie`, ``Non-Isometric Manifold Learning: Analysis and an Algorithm'', {\em 
International Conference on Machine Learning}, (\textbf{ICML}), 2007.

P.~Doll\'ar, V.~Rabaud and S.~Belongie, ``Learning to Traverse Image Manifolds'' , {\em Neural Information Processing 
Systems}, (\textbf{NIPS}), 2006. 

V.~Rabaud and S.~Belongie, ``Counting Crowded Moving Objects,'', {\em IEEE Conference on Computer Vision and Pattern 
Recognition}, (\textbf{CVPR}), 2006, pp. 705- 711, vol. 1.

P.~Doll\'ar, V.~Rabaud, G.~Cottrell and S.~Belongie, ``Behavior Recognition via Sparse Spatio-Temporal Features,'' {\em 
Joint International Workshop on Visual Surveillance and Performance Evaluation of Tracking and Surveillance}, 
(\textbf{VS-PETS}), 2005. 

S.~Belongie, K.~Branson, P.~Doll\'ar, and V.~Rabaud, ``Monitoring Animal Behavior in the Smart Vivarium,'' {\em 
International Conference on Methods and Techniques in Behavioral Research}, 2005.

V.~Rabaud and S.~Belongie, ``Big Little Icons,'' {\em IEEE Workshop on Computer Vision Applications for the Visually 
Impaired, in conjunction with CVPR}, (\textbf{CVAVI}), 2005.

K.\ Branson, V.\ Rabaud and S.\ Belongie, ``Three Brown Mice: See How They Run,''
{\em Joint International Workshop on Visual Surveillance and Performance Evaluation of Tracking and Surveillance}, 
(\textbf{VSPETS}), 2003, pp.\ 78-85. 

V.\ Rabaud and B.\ Deguine ``A Geometrical Approach To Determine Blackout Windows At Launch,''
AAS/AIAA Space Flight Mechanics Meeting, Ponce, Puerto Rico, (\textbf{AAS}), 2003, 03-187 


% Video
\sectiontitle{Logiciels}
{\em Surveillance Video Entertainment System}, (SVEN): logiciel de tracking en temps r\'eel de personnes, incluant un descripteur d'apparence, une d\'{e}tection de visage et une analyse d'expression.

{\em da Vinci Code}: logiciel rapide pour fusionner des images en une mosa\"{i}que haute r\'{e}solution.  L'accent a \'{e}t\'{e} mis sur l'efficacit\'{e} et la n\'{e}cessit\'{e} de faibles ressources.  Ce logiciel inclut un ajustement de faisceaux, les descripteurs SIFT et une calibration automatique.

{\em Vincent's Structure from Motion Toolbox for Matlab}: toolbox pour Matlab incluant plusieurs routines de reconstruction 3D (pour un objet rigide, non-rigide, ajustement de faisceaux, calcul d'orientation \dots).

% Video
\sectiontitle{Vid\'{e}o}
M.~Maschion, V.~Rabaud and S.~Belongie, {\em Computer Vision: Fact and Fiction},
Instructional DVD, 2005.

% Papers
\sectiontitle{Comp\'{e}tences}
{\em Syst\`{e}mes}: Windows, Unix, Solaris\\
{\em Langages}: C++, Python, Matlab, Javascript, Fortran \\
{\em Librairies}: OpenCV, ROS, Boost, OpenMP, TBB, GStreamer, PVM, MPI, GTK \\
{\em Langues}: Fran\c{c}ais (natif), Anglais (fluent), Espagnol (fluent), Portugais (d\'{e}butant), Italien 
(d\'{e}butant)

% Papers
\sectiontitle{Cours Appropri\'{e}s}
Computer Vision I (analyse d'image, radiom\'{e}trie, reconstruction 3D)\\
Computer Vision II (g\'{e}om\'{e}trie d\'{e}finie par plusieurs vues)\\
Computer Vision III (reconnaissance d'objet)\\
Apprentissage Automatique\\
Graphes\\

% References
\sectiontitle{References}

% \textbf{Prof. Sanjoy Dasgupta}\\
% University of California, San Diego – Computer Science \& Engineering\\
% CSE-EBU3B 4138; 9500 Gilman Dr.; \#0404 La Jolla, CA 92093-0404, USA\\
% dasgupta@cs.ucsd.edu

% \textbf{Maurizio Seracini}\\
% Editech\\
% Via Dei Bardi, 28\\
% 50125 Firenze (FI), Italy

\textbf{Prof. Serge Belongie}\\
Cornell University\\
111 Eighth Avenue \#302, New York, NY 10011, USA\\
sjb344@cornell.edu

\textbf{Dr. Gary Bradski}\\
Willow Garage, Inc. \\
68 Willow Road, Menlo Park, CA 94025, USA\\
gary@industrial-perception.com

\textbf{Dr. Kurt Konolige}\\
Industrial Perception, Inc. \\
911 Industrial Ave, Palo Alto, CA 94303, USA\\
kurt@industrial-perception.com

\textbf{Dr. Brian Gerkey}\\
Open Source Robotics Foundation\\
419 N Shoreline Blvd, Mountain View, CA 94043, USA\\
gerkey@osrfoundation.org

\textbf{Prof. David Kriegman}\\
University of California, San Diego – Computer Science \& Engineering\\
CSE-EBU3B 4120; 9500 Gilman Dr.; \#0404 La Jolla, CA 92093-0404, USA\\
kriegman@cs.ucsd.edu


\end{llist}

{\em Last update: \today}

\end{document}
